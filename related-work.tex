% !TeX root=main
% !TeX spellcheck=en_US

\section{Related Work}\label{sec:related}

\subsection{De-Identification}
De-identification is \iac{nlp} task that is related to \ac{ner}.
%
In 2006, 2014 and 2016, three shared tasks on de-identification were organized by the American i2b2 group.
%
The organizers performed manual pseudonymization on clinical records from a single source to create the datasets for each of the shared tasks~\citep{stubbs2015annotating}.

%
In the first two shared tasks, submitted approaches are based on hand-crafted rules, gazetteers, and machine learning methods such as boosting, \acp{crf}, and support vector machines.
%
The machine learning models use features such as lexical cues (e.g.\ \textit{``was the previous word `Dr'?''}), templates for phone numbers and addresses, part-of-speech tags, and gazetteer or dictionary information.
%
Similar approaches are also summarized in a survey~\citep{meystre2010automatic}.
%
In the 2016 shared task, a deep learning based system~\citep{liu2017identification} delivered the best results.

%
Up to the 2014 shared task, the organizers emphasized that it is unclear if a tool trained on the provided datasets will generalize to medical records from other sources~\citep{uzuner2007evaluating,stubbs2015automated}.
%
The 2016 shared task featured a sight-unseen track in which existing systems were evaluated on records from a new data source.
%
The best system achieved an \fone score of only $79\%$, proving that de-identification systems at the time were not able to deliver sufficient performance on completely new data~\citep{stubbs2017identification}.

%
\citet{dernoncourt2017identification} achieve state-of-the-art performance in de-identification with a deep learning model.
%
Their model achieves \fone scores of $97.85\%$ on the i2b2 2014 dataset and $99.23\%$ on their own, larger dataset.
%
It uses pre-trained word embeddings and task-specific character embeddings.
%
The output sequence is optimized with \iac{crf} layer.
%
After experimenting with their architecture, they conclude that the character embedding layer is more important to the model performance than the pre-trained word embeddings.
%
However, the word embeddings show benefits when compared to rule-based models e.g.\ when recognizing profession names that are in the same embedding space region but might not occur in a hand-crafted dictionary.
%
Using transfer learning from a model trained on a larger dataset, they were able to further improve their scores on the i2b2 2014 dataset~\citep{lee2017transfer}.
%
It is possible to reach an $98\%$ \fone score on the i2b2 dataset by using document position information as an additional input~\citep{zhao2018leveraging}, which will, however, most likely deteriorate generalization performance to instances of medical text with different structures.
%
In a later work, \citeauthor{dernoncourt2017identification} use the same architecture for a generic \ac{ner} model~\citep{dernoncourt2017neuroner}, which underlines the similarity between the de-identification and \ac{ner} tasks.

%
De-identification can be interpreted as a preprocessing step for further information extraction because only de-identified data is allowed to be shared.
%
Such information extraction tasks include \ac{ner} of condition and treatment names~\citep{uzuner2010extracting,pradhan2014semeval} or the classification of relationships between conditions, tests, and treatments~\citep{uzuner2011challenge}.
%
A survey~\citep{meystre2008extracting} summarizes information extraction approaches at the time.
%
In 2012, the free-text part of electronic health records was still considered an underused source of data~\citep{jensen2012mining}.

%
In \citeauthor{dernoncourt2017identification}'s models, Wikipedia-pre-trained GloVe word embeddings~\citep{pennington2014glove} perform better than embeddings that were trained on the i2b2 dataset.
%
In \ac{ner} tasks focusing on medical entities, training embeddings on unlabeled medical text works well~\citep{wu2015study}, possibly because they cover the medical domain vocabulary better.

\subsection{Adversarial Representation Learning}
\begin{itemize}
    \item Adversarial learning in neural nets: \citet{goodfellow2014generative}
    \item \Ac{dann} architecture \citep{ganin2016domain}
    \item Uses in anonymization / demographics independent representation \citet{elazar2018adversarial,li2018towards}
    \item \citet{elazar2018adversarial} cautionary conclusion (continued adversary training beats the representation)
    \item \citet{feutry2018learning} three-step learning procedure
\end{itemize}
