% !TeX root=main
% !TeX spellcheck=en_US

\section{Related Work}\label{sec:related}

\subsection{De-Identification}
De-identification is \iac{nlp} sequence tagging task.
%
In 2006, 2014 and 2016, three shared tasks on de-identification were organized by the American i2b2 group.
%
The organizers performed manual pseudonymization on clinical records from a single source to create the datasets for each of the shared tasks \citep{stubbs2015annotating}.
%
A binary \ac{hipaa} token \fone score of $95\%$ has been suggested as a target score for reasonable de-identification systems \citep{stubbs2015automated}.

%
Up to the 2014 shared task, the organizers emphasized that it is unclear if a tool trained on the provided datasets will generalize to medical records from other sources \citep{uzuner2007evaluating,stubbs2015automated}.
%
The 2016 shared task featured a sight-unseen track in which existing systems were evaluated on records from a new data source.
%
The best system achieved an \fone score of only $79\%$, proving that de-identification systems at the time were not able to deliver sufficient performance on completely new data \citep{stubbs2017identification}.

%
\citet{dernoncourt2017identification} achieve state-of-the-art performance in de-identification with \iac{lstm} model.
%
Their model achieves \fone scores of $97.85\%$ on the i2b2 2014 dataset and $99.23\%$ on their own, larger dataset.
%
It uses pre-trained GloVe word embeddings and task-specific character embeddings.
%
The output sequence is optimized with \iac{crf} layer.
%
After experimenting with their architecture, they conclude that the character embedding layer is more important to the model performance than the pre-trained word embeddings.
%
Using transfer learning from a model trained on a larger dataset, they were able to further improve their scores on the i2b2 2014 dataset \citep{lee2017transfer}.
%
It is possible to reach an $98\%$ \fone score on the i2b2 dataset by using document position information as an additional input \citep{zhao2018leveraging}, which will, however, most likely deteriorate generalization performance to instances of medical text with different structures.

\subsection{Adversarial Representation Learning}
\begin{itemize}
    \item Adversarial learning in neural nets: \citet{goodfellow2014generative}
    \item \Ac{dann} architecture \citep{ganin2016domain}
    \item Uses in anonymization / demographics independent representation \citet{elazar2018adversarial,li2018towards}
    \item \citet{elazar2018adversarial} cautionary conclusion (continued adversary training beats the representation)
    \item \citet{feutry2018learning} three-step learning procedure
\end{itemize}
