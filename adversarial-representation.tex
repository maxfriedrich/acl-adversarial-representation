% !TeX root=main
% !TeX spellcheck=en_US

\section{Adversarial Representation}\label{sec:adversarial-representation}
%
The previously discussed approaches are highly dependent on manually tuned representation parameters like the amount of noise and neighbor space, that, if set incorrectly, may allow for easy re-identification.
%
In this experiment, we evaluate an adversarial learning based approach that automatically tunes the representation parameters to protect against adversary models.

\begin{description}
    \item[Architecture]
    %
    Our approach uses a model that is composed of three components: a representation model, a de-identification model, and an adversary.
    %
    An overview of the architecture is shown in \cref{fig:adversarial-model}.
    %
    The representation model maps a sequence of word embeddings to an intermediate vector representation sequence.
    %
    The de-identification model receives this representation sequence as an input instead of the original embedding sequence.
    %
    It retains the casing feature as an additional input.
    %
    As before, the de-identification model outputs a sequence of class probabilities.
    %
    The representation is also used as an input to the adversary that tries to infer information about the original embedding sequence.
    
    \begin{figure}
        \centering
        \begin{tikzpicture}[node distance=1.4cm,font=\small]
\tikzset{every node/.style={inner sep=1mm, outer sep=0mm, line width=0mm}}

\tikzstyle{token}=[rectangle,draw=black,fill=white,semithick,text width=1cm, minimum width=1.2cm, minimum height=5mm, text height=1.5ex,text depth=.25ex]
\tikzstyle{model}=[rounded rectangle,draw=black,fill=white,semithick, minimum width=3cm, minimum height=8mm, text height=1.5ex,text depth=.25ex, inner sep=1.5mm]
\tikzstyle{dots} = []
\tikzstyle{vector} = [draw, shape=rectangle, fill=white, semithick, minimum width=1.2cm, minimum height=3mm]
\tikzstyle{half vector} = [shape=rectangle, semithick, minimum width=1.2cm, minimum height=3mm] % without drawn border
\tikzstyle{two thirds vector} = [draw, shape=rectangle, fill=white, semithick, minimum width=0.9cm, minimum height=3mm]
\tikzstyle{quarter vector} = [draw, shape=rectangle, fill=white, semithick, minimum width=3mm, minimum height=3mm]
\tikzstyle{pre}=[<-,semithick, >=latex]
\tikzstyle{post}=[->, semithick, >=latex]

\node[token] (mr input){Mr.};
\node[token, right of=mr input] (smith input) {Smith};
\node[token, right of=smith input] (was input) {was};
\node[dots, right=1mm of was input] (input dots) {$\cdots$};
    
\node[vector, above=8mm of mr input] (mr embedding) {};
\node[vector, right of=mr embedding] (smith embedding) {};
\node[vector, right of=smith embedding] (was embedding) {};
\node[dots, right=1mm of was embedding] (embedding dots) {$\cdots$};

\begin{scope}[on background layer]
    \node (input box) [draw,fill=black!5,fit=(mr input) (input dots), inner sep=1.5mm] {};
    \node (feature box) [draw,fill=black!5,fit=(mr embedding) (embedding dots), inner sep=1.5mm] {};
\end{scope}

\node[model,above=5mm of feature box] (representation model) {Representation Model};

\node[vector, above=2.2cm of mr embedding] (mr representation) {};
\node[vector, right of=mr representation] (smith representation) {};
\node[vector, right of=smith representation] (was representation) {};
\node[dots, right=1mm of was representation] (representation dots) {$\cdots$};

\begin{scope}[on background layer]
    \node (representation box) [draw,fill=black!5,fit=(mr representation) (representation dots), inner sep=1.5mm] {};
\end{scope}

\node[model,above left=8mm and -2cm of representation box] (deid model) {De-Identification Model};

\node[two thirds vector, above left=2.5cm and 0.5cm of mr representation] (mr output) {};
\node[two thirds vector, right=3mm of mr output] (smith output) {};
\node[two thirds vector, right= 3mm of smith output] (was output) {};
\node[dots, right=3mm of was output] (output dots) {$\cdots$};    

\begin{scope}[on background layer]
 \node (output box) [draw,fill=black!5,fit=(mr output) (output dots), inner sep=1.5mm] {};
\end{scope}

\node[model,above right=8mm and -1.5cm of representation box] (adversary model) {Adversary Model};

\node[quarter vector, above right=2.5cm and 4mm of was representation] (adversary output) {};

\begin{scope}[on background layer]
    \node (adversary output box) [draw,fill=black!5,fit=(adversary output), inner sep=1.5mm] {};
\end{scope}


% vector squares
\foreach \i in {mr embedding, smith embedding, was embedding, mr representation, smith representation, was representation} {
    \draw[semithick] (\i.north west) rectangle ($(\i.north west) + (3mm, -3mm)$);
    \draw[semithick] (\i.north west) rectangle ($(\i.north west) + (6mm, -3mm)$);
    \draw[semithick] (\i.north west) rectangle ($(\i.north west) + (9mm, -3mm)$);    
};

\foreach \i in {mr output, smith output, was output} {
    \draw[semithick] (\i.north west) rectangle ($(\i.north west) + (3mm, -3mm)$);
    \draw[semithick] (\i.north west) rectangle ($(\i.north west) + (6mm, -3mm)$);
};

\path[post] (input box) edge (feature box);
\path[post] (feature box) edge (representation model);
\path[post] (representation model) edge (representation box);

\path[post] (representation box) edge (deid model);
\path[post] (deid model) edge (output box);

\path[post] (representation box) edge (adversary model);
\path[post,dotted] (feature box.east) edge[bend right=30] (adversary model);
\path[post,dotted] (representation box.north east) edge (adversary model);
\path[post] (adversary model) edge (adversary output box);

\node[anchor=east, left=1mm of input box] {Tokens};
\node[anchor=east, left=1mm of feature box] {Emb.};
\node[anchor=east, left=1mm of representation box] {Represent.};
\node[anchor=center, above=1mm of output box] {De-identification output};
\node[anchor=center, above=1mm of adversary output box] {Adversary output};

\end{tikzpicture}
        \caption[Adversarial model architecture]{%
            Simplified visualization of the adversarial model architecture.
            %
            Sequences of squares denote real-valued vectors, dotted arrows represent possible additional real or fake inputs to the adversary.
            %
            The casing feature that is provided as a second input to the de-identification model is omitted for legibility.}\label{fig:adversarial-model}
    \end{figure}
    
    \item[Representations]
    %
    We evaluate two types of representation models: a feedforward and \iac{lstm} model.
    %
    Both apply Gaussian noise with zero mean and trainable standard deviations to their inputs and outputs.
    %
    The models learn a standard deviation for each of the input and output dimensions.
    
    %
    We try different representation sizes to explore the trade-off between de-identification and adversary performances.
    %
    In contrast to the approaches from \cref{sec:perturbing} that only perturb \ac{phi} tokens, the representation models in this approach process all tokens to represent them in a new embedding space.
    
    \item[Adversaries]
    %
    In existing gradient reversal approaches \citep{ganin2016domain,feutry2018learning,elazar2018adversarial}, the learned representation is invariant to some attribute of the input.
    %
    Similarly, our representation should be invariant to small input changes, like a single token being replaced with a neighbor in the embedding space.
    %
    The number of neighbors $N$ controls the privacy properties of the representation.
    
    %
    Additionally, we need our representation to contain a random element because we want to share the output representations as well as the representation model itself.
    %
    An attacker should not be able to create a lookup table of representations for exact sentences, i.e.\ the representation must be immune to known-plaintext attacks.
    
    %
    To achieve these goals, we use two adversaries that are trained for the following tasks:
    \begin{enumerate}
        \item Given a representation and an embedding sequence, decide if they were obtained from the same sentence.
        \item Given two representation sequences (and their cosine similarities), decide if they were obtained from the same sentence.
    \end{enumerate}
    
    %
    \Cref{fig:adversaries} shows the two adversaries with their respective inputs.
    %
    The first adversary's objective is a discriminatory formulation of an inverse representation model and causes representations for similar inputs (replacing any protected token with one of its $N$ neighbors) to be indistinguishable.
    %
    The second adversary's objective causes repeated representation computations for the same sentence to differ by a high enough degree to make it impossible to build a lookup table of representations.
    %
    We obtain the representation sequences for the second adversary from copies of the representation model with shared weights.
    %
    We generate real and fake pairs for adversarial training using the automatic pseudonymization approach presented in \cref{sec:auto-pseudo}, limiting the number of replaced tokens to one per sentence.
    
    %
    The adversaries are implemented as bidirectional \ac{lstm} models.
    %
    We confirmed that bidirectional \ac{lstm} models are able to learn the adversarial tasks on randomly generated data and raw word embeddings in a preliminary experiment.
    %
    To use the two adversaries in our architecture, we average their outputs.
    
    \item[Training]
    %
    We evaluate two training procedures: \ac{dann} training~\citep{ganin2016domain} and the alternating approach by \citet{feutry2018learning}.
    
    %
    In \ac{dann} training, the three components are trained conjointly, optimizing the sum of losses.
    %
    Training the de-identification model modifies the representation model weights to generate a more meaningful representation for de-identification.
    %
    The adversary gradient is reversed with a gradient reversal layer between the adversary and the representation model in the backward pass, causing the representation to become less meaningful for the adversary.
    
    %
    The training procedure by \citet{feutry2018learning} is shown in \cref{fig:feutry-training}.
    %
    It is composed of three sequential phases:
    %
    \begin{enumerate}
        \item The de-identification and representation models are pre-trained together, optimizing the de-identification loss $L_{\text{deid}}$.
        \item The representation model is frozen and the adversary is pre-trained, optimizing the adversarial loss $L_{\text{adv}}$.
        \item In alternation, for one epoch each:
        \begin{enumerate}
            \item The representation is frozen and both de-identification model and adversary are trained, optimizing their respective losses $L_{\text{deid}}$ and $L_{\text{adv}}$.
            \item The de-identification model and adversary are frozen and the representation is trained, optimizing the combined loss $L_{\text{repr}} = L_{\text{deid}} + \lambda \abs{L_{\text{adv}} - L_{\text{random}}}$. \label{item:repr-training}
        \end{enumerate}
    \end{enumerate}
    
    \begin{figure}
        \centering
        \begin{tikzpicture}[node distance=2.5cm,font=\small, sibling distance=1.2cm, level distance=1.8cm, grow=up, edge from parent/.style = {->, >=latex, draw}]
\tikzset{every node/.style={inner sep=1mm, outer sep=0mm, line width=0mm}}

\tikzstyle{model}=[rounded rectangle,draw=black,fill=white, minimum width=1.2cm, minimum height=8mm, semithick]
\tikzstyle{train}=[ultra thick]
\tikzstyle{label}=[text height=0.75ex,text depth=0ex]
\tikzstyle{dots} = []
\tikzstyle{pre}=[<-,semithick, >=latex]
\tikzstyle{post}=[->, semithick, >=latex]

\node[model, train] (a) {}
    child {node[model] (a adv) {}}
    child {node[model, train] (a deid) {}};

\node[label, below= 2mm of a] (a label) {1.}; 

\node[model, right=of a] (b) {}
    child {node[model, train] (b adv) {}}
    child {node[model] (b deid) {}};

\node[label, below=2mm of b] (b label) {2.}; 

\node[model, right=of b] (c) {}
    child {node[model, train] (c adv) {}}
    child {node[model, train] (c deid) {}};
    
\node[label, below=2mm of c] (c label) {3.$\,$a)}; 

\node[model, right= of c, train] (d) {}
    child {node[model] (d adv) {}}
    child {node[model] (d deid) {}};
    
\node[label, below=2mm of d] (d label) {3.$\,$b)}; 
    
\begin{scope}[on background layer]
    \node (a box) [draw,fill=black!5,fit=(a) (a label) (a adv) (a deid), inner sep=2mm] {};
    \node (b box) [draw,fill=black!5,fit=(b) (b label) (b adv) (b deid), inner sep=2mm] {};
    \node (c box) [draw,fill=black!5,fit=(c) (c label) (c adv) (c deid), inner sep=2mm] {};
    \node (d box) [draw,fill=black!5,fit=(d) (d label) (d adv) (d deid), inner sep=2mm] {};
\end{scope}

\path[post] (a box) edge (b box);
\path[post] (b box) edge (c box);
\path[post] (c box) edge[bend right=10] (d box);
\path[post] (d box) edge[bend right=10] (c box);

\end{tikzpicture}
        \caption[Adversarial training procedure]{%
            Visualization of \citeauthor{feutry2018learning}'s training procedure.
            %
            The adversarial model layout follows \cref{fig:adversarial-model}: the representation model is at the bottom, the left branch is the de-identification model and the right branch is the adversary.
            %
            In each step, the thick components are trained while the thin components are frozen.
            %
            Steps 1 and 2 are trained until stable.
            %
            Then training alternates between one epoch and step 3a and one epoch of step 3b.
        }\label{fig:feutry-training}
    \end{figure}
    
    %
    In the first two phases, we monitor the respective validation losses for early stopping to decide at which point the training should move on to the next phase.
    %
    The alternating steps in the third phase each last one training epoch.
    %
    We determine the early stopping epoch using only the combined validation loss (\cref{item:repr-training}).
    
    %
    Gradient reversal is achieved by optimizing the combined representation loss while the adversary weights are frozen.
    %
    The combined loss is motivated by the fact that the adversary performance should be the same as a random guessing model, which is a lower bound for anonymization~\citep{feutry2018learning}.
    %
    The term $\abs{L_{\text{adv}} - L_{\text{random}}}$ approaches $0$ when the adversary performance approaches random guessing\footnote{In the case of binary classification: $L_{\text{random}} = -\log \frac{1}{2} \approx 0.6931$.}.
    %
    $\lambda$ is a weighting factor for the two losses; we select $\lambda=1$.
    
    \item[Application]
    %
    To apply the model in practice, a central model provider would train the three parts of the model on an initial \ac{phi}-annotated dataset, e.g.\ the i2b2 2014 data.
    %
    This initial training should confirm that the learned representation allows training a de-identification model while being robust to the adversaries.
    %
    The model provider would then publish the representation model along with their choice of pre-trained word embeddings.
    %
    Medical institutions would use the representation model to transform their \ac{phi}-labeled data into a private representation, which is then sent back to the central model provider with the respective labels.
    %
    This transformation replaces the manual document-coherent pseudonymization that is typically performed to share training data for de-identification.
    
    %
    The model provider would then update the existing de-identification model or train a new model using all available representation data.
    %
    Periodically, the pipeline of representation model (possibly in a version without additive noise) and de-identification model would be published so it can be used by medical institutions on their unlabeled data.
\end{description}