% !TeX root=main
% !TeX spellcheck=en_US

\section{Automatic Pseudonymization}\label{sec:automatic-pseudonymization}
%
As a further perturbation method, we evaluate a naive automatic approximation of pseudonymization by substitution.
%
Before training, we randomly move all \ac{phi} tokens to the position of one of a fixed number $N$ of their neighbors in an embedding space, as determined by cosine distance in a pre-computed embedding matrix.

%
In GloVe, only tokens that exist in the pre-computed embedding matrix are moved to their neighbors.
%
The unknown token is not modified as it does not contain any information (except that the token in question is not part of the precomputed matrix).

%
We evaluate the privacy properties of the approach using a bidirectional \ac{lstm} adversary with a single output unit.
%
It is trained on the tasks of distinguishing pairs of:
\begin{itemize}
    \item a pseudonymized sequence and the original sequence
    \item a pseudonymized sequence and a minimally modified original sequence (with only one occurrence of \ac{phi} moved to one of its neighbors).
\end{itemize}
%
For the minimally modified original sequence, we keep the number of neighbors fixed at $N=5$ to avoid augmented sentences being too unrealistic.
