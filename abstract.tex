% !TeX root=main
% !TeX spellcheck=en_US

\begin{abstract}
%
De-identification is the task of detecting \ac{phi} in medical text.
%
It is a critical step in sanitizing \acp{ehr} to be shared for research.
%
Automatic de-identification classifiers can significantly speed up the sanitization process.
%
However, obtaining a large and diverse dataset to train such a classifier that works well across many types of medical text poses a challenge as privacy laws prohibit sharing of raw medical records.
%
We introduce a method to create privacy-preserving shareable representations of medical text (i.e.\ they contain no \ac{phi}) that does not require expensive manual pseudonymization. 
%
These representations can be shared between organizations to create unified datasets for training de-identification models.
%
Our representation allows training a simple \acs{lstm}-\acs{crf} de-identification model to an \fone score of $97.4\%$, which is comparable to a strong non-private baseline.
%
A robust, widely available de-identification classifier based on our representation could potentially enable medical studies for which de-identification would otherwise be too costly.
\end{abstract}
