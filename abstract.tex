% !TeX root=main
% !TeX spellcheck=en_US

\begin{abstract}
%
De-identification is the task of detecting \ac{phi} in medical text.
%
Manual de-identification is often necessary for sharing medical data and automatic classifiers can significantly
speed up the de-identification process.
%
However, obtaining a large and diverse dataset to train such a classifier poses a challenge as privacy laws prohibit sharing of raw medical records.
%
We introduce a method to create shareable representations of text
(i.e.\ they contain no \ac{phi}) which does not require expensive
manual pseudonymization.  These representations can be shared between
organizations to create unified datasets for training de-identification
models.
%
Our representation allows training a simple \acs{lstm}-\acs{crf} de-identification model to an \fone score of $97.4\%$, which is comparable to a strong non-private baseline.
\end{abstract}

%%% Local Variables:
%%% mode: latex
%%% TeX-master: "main"
%%% End:
