% !TeX root=main
% !TeX spellcheck=en_US

\section{Introduction}\label{sec:introduction}
%
In addition to structured medical data, electronic health records contain free-text patient notes that are a rich source of information \citep{jensen2012mining}.
%
Due to privacy and data protection laws, medical records can only be shared and used for research if they are sanitized.
%
De-identification is the task of finding and labeling \ac{phi} in medical text for sanitization.
%
\Ac{phi} includes potentially identifying information such as names, professions, geographic identifiers, dates, and account numbers.
%
The American \ac{hipaa} defines 18 categories of \ac{phi}.

% TODO Why is it important? (enable large-scale studies)

% TODO de-identification as a sequence tagging task like NER

%
Trying to train a software tool for automatic de-identification leads to a ``chicken and egg problem''~\citep{uzuner2007evaluating}: without a comprehensive training set, an automatic de-identification tool cannot be developed, but without such a tool, it is difficult to share de-identified medical text for research (including for training the tool itself).
%
The standard method of data protection compliant sharing of training data for a de-identification tool requires humans to pseudonymize protected information with substitutes (replacing e.g.\ every person name with a different person name and every date with a different date) in a document-coherent way \cite{uzuner2007evaluating}.

% TODO including applying spelling mistakes to the substitutes etc

% Why is manual pseudonymization not so great? (time-consuming, coherency)
%
Today, a pseudonymized dataset for de-identification from a single source is publicly available \citep{stubbs2015annotating}.
%
However, de-identification tools trained on the dataset are too specific for the concrete data and do not generalize well to data from other sources~\citep{stubbs2017identification}.
%
If a medical institution instead decides to train a de-identification tool on their raw text data, it is conceivable that the tool would contain traces of the \ac{phi} it was trained with, making it possible for an attacker to recover parts of the training data if the tool itself is shared.
%
To achieve a universal de-identification tool, many medical institutions would have to pool their data.
%
Preparing this data for sharing using the document-coherent pseudonymization approach requires large human effort \citep{dernoncourt2017identification}.

% Why is sharing training data the best solution for a de-identification model?
% TODO automatic pseudonymization as a requirement/precursor for the learned representation instead of describing them as ``two approaches''
%
We introduce two representation approaches to privacy-preserving sharing of medical text that allow training a de-identification tool: an automatic pseudonymization and an adversarially learned private representation.
%
Our approaches still requires humans to annotate \ac{phi} (as this is the training data for the task) but the pseudonymization step is performed by the transformation to the representations.
%
A tool trained on our representations could easily be made publicly available because its parameters cannot contain any protected data, as it is never trained on raw text.
%
Simplifying the de-identification procedure may enable large-scale medical studies that are otherwise too costly.

% TODO present results
% use acronym short forms here (\acs{…}) and explain them later
We train a basic \acs{lstm}-\acs{crf} de-identification model on raw and automatically pseudonymized text as well as on our adversarial private representation.
%

